%@+leo-ver=4-thin
%@+node:paran.20140515173914.6982:@shadow ./fmpp/Tex/Thesis/Chapters/versionControl.tex
%@@color
%@@language latex
\chapter{Version Control Systems}




%@<<Introduction>>
%@+node:paran.20140515173914.6996:<<Introduction>>
were written with two issues in mind
1) to keep revisions of code
2) to allow collaboration on code
%@-node:paran.20140515173914.6996:<<Introduction>>
%@nl
%@<<Dawn of Time>>
%@+node:paran.20140515173914.6999:<<Dawn of Time>>
/section{Before version control}
Before version control the only way to keep a revision was to copy the code and save it into a different file on the file system
%@nonl
%@-node:paran.20140515173914.6999:<<Dawn of Time>>
%@nl
%@<<RCS>>
%@+node:paran.20140515173914.6995:<<RCS>>

/section{Revision Control System (RCS)}
RCS was one of the oringinal versioning systems 
It allowed a small amount of collaboration by locking files that had been checked out.
This ensured that the checked out file could only be used by one person at a time.
Instead of keeping multiple copies of the file that had changed it kept the original file and any of the subsequent change sets.
this 
%@nonl
%@-node:paran.20140515173914.6995:<<RCS>>
%@nl
%@<<Merging>>
%@+node:paran.20140515173914.6994:<<Merging>>
\section{Merging two documents}
automatic manual
%@nonl
%@-node:paran.20140515173914.6994:<<Merging>>
%@nl
%@<<Centralised Version control>>
%@+node:paran.20140515173914.6993:<<Centralised Version control>>
/section{centralised version control}
Centralised version control
still had the model of a central repository everyone needed to containt
similar to RCS in that you could still check the file out but it was not as necessary to lock the file
it was possible to look at the differences between the files and then automatically merge when it was possible
%@nonl
%@-node:paran.20140515173914.6993:<<Centralised Version control>>
%@nl
%@<<Distributed Version control>>
%@+node:paran.20140515173914.6992:<<Distributed Version control>>
%@<<Git>>
%@+node:paran.20140515173914.6991:<<Git>>
\chapter{Using git}
Git is a repository which is usually used mostly for software development.
A repository is a way to keep old revisions of a document backed up so that if you ever need to revist a previous revision you can always access it.
Repositories also allow people to work on the same document at the same time.
This is done by combining all the changes to a document in a process called merging. 
In order to merge, any change an editor makes needs to be recorded and compared against the changes made by other editors.
If it is possible for those changes to co-exist then the changes will be made.
An example of changes that are considered to be able to co-exist is if all editors change a different part of the document.
If it is not possible for those changes to co-exist then there is a \"merge conflict\".
An example of a merge conflict is if any two changes on the document overlap with different values.
Before any merging can be done all of the changes need to be determined
%@nonl
%@-node:paran.20140515173914.6991:<<Git>>
%@nl
%@nonl
%@-node:paran.20140515173914.6992:<<Distributed Version control>>
%@nl
 
 


%@-node:paran.20140515173914.6982:@shadow ./fmpp/Tex/Thesis/Chapters/versionControl.tex
%@-leo
