%@+leo-ver=4-thin
%@+node:paran.20140515173914.7008:@shadow ./fmpp/Tex/Thesis/Chapters/Lcs.tex
%@@color
%@@language latex
\chapter{Longest common subsequence}

%@<<Longest common subsequence>>
%@+node:paran.20140515173914.7009:<<Longest common subsequence>>
\section{Longest Common Subsequence}
One method of discovering what has changed is to find the longest common subseqeunce (LCS).
A simplified example of finding the longest common subsequence is:

Imagine we have have two similar sentances that we want to compare with each other.  
We would like to know what is the same and what is different.
A longest common subsequnce for the sentances would contain a list of all the characters that are the same and in the same order
So for the following sentances

\begin{verbatim}

\"The quick brown fox jumps over the lazy dog\"

\"The rapid brown fox vaults the lazy dog\"

\end{verbatim}
A longest common subsequence would be
\begin{verbatim}
\"The \",\"i\",\" brown fox \",\"v\",\" the lazy dog\"
\end{verbatim}
The letters that are missing from the LCS differ between the sentances.
%@nonl
%@-node:paran.20140515173914.7009:<<Longest common subsequence>>
%@nl

%@<<Git difference strategies>>
%@+node:paran.20140515173914.7010:<<Git difference strategies>>
\section{Git difference strategies}
Git uses the following algorithms to find the LCS.

%@<<Myers>>
%@+node:paran.20140515173914.7011:<<Myers>>
\subsection{Myers}

%@-node:paran.20140515173914.7011:<<Myers>>
%@nl

%@<<Patience>>
%@+node:paran.20140515173914.7012:<<Patience>>
\subsection{Patience}

%@+at
% fron the patience game
% need references
%@-at
%@@c 
%@-node:paran.20140515173914.7012:<<Patience>>
%@nl

%@<<Histogram>>
%@+node:paran.20140515173914.7013:<<Histogram>>
\subsection{Histogram}
%@nonl
%@-node:paran.20140515173914.7013:<<Histogram>>
%@nl
%@-node:paran.20140515173914.7010:<<Git difference strategies>>
%@nl

%@<<The problem with LCS>>
%@+node:paran.20140515173914.7014:<<The problem with LCS>>
\section{The problem with longest common subsequence}
There is still a problem with longest common subsequence. It does not notice changes of order in a document.  For example if we were to take the following two sentances:

\begin{verbatim}

\"The quick brown fox jumps over the lazy dog\"

\"The lazy brown dog jumps over the quick fox\"

\end{verbatim}

The longest common subsequence of this would be

\begin{verbatim}
\"The \",\" brown \",\"o\",\" jumps over the \",\"o\"
\end{verbatim}

Without further analysing the changes it is possible to conclude that instead of swapping certain words that:

\begin{verbatim}
\"quick\" transforms into \"lazy\"
\"f\" transforms into \"d\"
\"x\" transforms into \"g\"
\"lazy d\" transforms into \"quick f\"
\"g\" transforms into \"x\"
\end{verbatim}

What this thesis aims to do is to more accurately portray these changes.
In order to do this we require some information about the structure of the document.
For the above example if the computer was aware that the sentence was structured into words rather than characters the result would have been slightly different.

\begin{verbatim}
\"The \",?,\" brown \",?,\"jumps \", \"over \", \"the \",?,?
\end{verbatim}

in this situation in becomes easier to recognise that words have been swapped by comparing each of the changes with each other.  

\begin{verbatim}
\"quick \" transforms into \"lazy \" matches \"lazy\" transforms into \"quick\"
\"fox \" transforms into \"dog\" matches \"dog\" transforms into \"fox\" 
\end{verbatim}

The English language is also far too complex to notice anything that is more basic than a word for word swap.
There are words and sentences that have similar meanings but are spelt and structured differently.
%@nonl
%@-node:paran.20140515173914.7014:<<The problem with LCS>>
%@nl

%@<<How diffs use LCS>>
%@+node:paran.20140515173914.7015:<<How diffs use LCS>>
\section{How most difference tools use LCS}
The above is a simplified illustration of how general LCS works.
How most difference tools use LCS is quite.
Most difference tools rather than comparing on a character by character difference compares line of text with each other.

%@+at
% Remember to say something white space
% and regular expression differences
% 
%@-at
%@@c
%@nonl
%@-node:paran.20140515173914.7015:<<How diffs use LCS>>
%@nl
 
 


%@-node:paran.20140515173914.7008:@shadow ./fmpp/Tex/Thesis/Chapters/Lcs.tex
%@-leo
