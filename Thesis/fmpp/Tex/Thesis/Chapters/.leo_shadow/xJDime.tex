%@+leo-ver=4-thin
%@+node:paran.20140501082248.1919:@shadow ./fmpp/Tex/Thesis/Chapters/JDime.tex
%@@color
%@@language latex

\chapter{JDime}
Part of the inspiration for this tool come from JDime
JDime is useful for quickly merging two different change sets. 

%@<<What Jdime can be used for>>
%@+node:paran.20140516082812.1938:<<What Jdime can be used for>>
%@-node:paran.20140516082812.1938:<<What Jdime can be used for>>
%@nl

JDime instead of testing against a source code repository test against files in the system under the base, left and right directories.
While this may be useful in quickly being able to show what JDime is able to achieve it requires that the inputs need to be previously extracted from a repository into the file system.

Before doing any calculations, Jdime runs a regular text merge over the source code.  
If the regular text merge has conflicts then JDime determines if sections of the source code need to be in a particular order or could be in any order.
What then happens depends on if order is required in the section of code JDime is examining.


In order to examine how Jdime works a test handler was written.
The test handler sets up some java source code to be used by JDime in base, left and right directories.
In order to test JDime it is necessary to cause the initial text comparison to incorrectly label as a conflict.
A way to get a lot of text conflicts between two pieces of code that are equivalent when they run is to change the order of the methods.
Although the methods are in different order the programs are still \"functionally equivalent\".
The order of the methods where scrambled in files in the base, left and right directories.





 







%@-node:paran.20140501082248.1919:@shadow ./fmpp/Tex/Thesis/Chapters/JDime.tex
%@-leo
