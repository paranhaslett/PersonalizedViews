
\chapter{Experimental Results}

The data sets in table ~\ref{tab:DataSets}are complete git repositories which have been examined to discover the proportion of changes to behavior compared to ascetic changes that do not affect the behaviour of the program. 
\begin{table}
  \caption{Complete git repositories that are tested}
  \label{tab:DataSets}
  \begin{center}
    \begin{tabular}{|c|c|c|}
       Jasm & this a a Java bytecode assembler 
       
       written for use with the Whiley programming language. & https://github.com/Whiley/Jasm\\
      foo & bar \\
    \end{tabular}
  \end{center}
\end{table}

\begin{itemize}
  \item Jasm - this a a Java bytecode assembler written for use with the Whiley programming language. https://github.com/Whiley/Jasm
  \item Jpp - this is a preproccessor for Java based on Bash shell script. https://github.com/maandree/jpp
  \item AST Java - a small parser wriiten to transform Java into an AST. https://github.com/klangner/ast-java
  \item Java Object Diff - allows two Java objects to be compared https://github.com/SQiShER/java-object-diff
  \item DiffJ - a diff tool which in addition to ignoring white space ignores changes of ordering in package names for a Java file https://github.com/jpace/diffj
\end{itemize}

% Possible datasets
% <A HREF="https://github.com/clojure/clojure" >clojure/clojure · GitHub</A>
% A lisp like language written in Java and targeting the JVM
% <A HREF="https://github.com/rzwitserloot/lombok" >rzwitserloot/lombok · 
% GitHub</A>
% A Java library that uses annotations to simplify commonly written code
% <A HREF="https://github.com/maandree/jpp" >maandree/jpp · GitHub</A>
% A preprocessor for Java
% <A HREF="https://github.com/klangner/ast-java">klangner/ast-java · 
% GitHub</A>
% parses and represents Java as an AST
% <A HREF="https://github.com/antlr/antlr4">antlr/antlr4 · GitHub</A>
% A parser for any language you have language definition for
% 
% JaCoCo Java Code Coverage Library
% JaCoCo is a free Java code coverage library distributed under the Eclipse 
% Public License. Check http://www.eclemma.org/jacoco for downloads, 
% documentation and feedback.
% https://github.com/jacoco/jacoco
% 
% Java Object Diff
% https://github.com/SQiShER/java-object-diff
% 
% java-object-diff is a simple, yet powerful library to find differences 
% between Java objects. It takes two objects and generates a tree structure 
% that represents any differences between the objects and their children. This 
% tree can then be traversed to extract more information or apply changes to 
% the underlying data structures.
% 
% Build Status
% 
% Features
% 
% Generates an easily traversable tree structure to analyze and modify with 
% surgical precision
% Detects whether a value or item has been added, removed or changed and shows 
% the changes
% Allows to manipulate the underlying objects directly through the generated 
% tree nodes
% Works with almost any kind of object (Beans, Lists, Maps, Primitives, 
% Strings, etc.)
% Properties can be marked with category tags to easily filter for specific 
% subsets
% No configuration needed (but possible)
% No runtime dependencies except for SLF4J
% When needed, it leaves it up to you, whether you want to use declarative 
% configuration or annotations
% Getting Started
% 
% To learn how to use Java Object Diff, please have a look at the Starter 
% Guide.
% 
% Why would you need this?
% 
% Sometimes you need to figure out, how one version of an object differs from 
% another one. One of the simplest solutions that'll cross your mind is most 
% certainly to use reflection to scan the object for fields or getters and use 
% them to compare the values of the different object instances. In many cases 
% this is a perfectly valid strategy and the way to go. After all, we want to 
% keep things simple, don't we?
% 
% However, there are some cases that can increase the complexity dramatically. 
% What if you need to find differences in collections or maps? What if you 
% have to deal with nested objects that also need to be compared on a 
% per-property basis? Or even worse: what if you need to merge such objects?
% 
% You suddenly realize that you need to scan the objects recursively, figure 
% out which collection items have been added, removed or changed; find a way 
% to return your results in a way that allows you to easily access the 
% information you are looking for and provide accessors to apply changes.
% 
% While all this isn't exactly rocket science, it is complex enough to add 
% quite a lot of extra code to your project. Code that needs to be tested and 
% maintained. Since the best code is the code you didn't write, this library 
% aims to help you with all things related to diffing and merging of Java 
% objects by providing a robust foundation and a simple, yet powerful API.
% 
% This library will hide all the complexities of deep object comparison behind 
% one line of code:
% 
% Node root = ObjectDifferFactory.getInstance().compare(workingObject, 
% baseObject);
% This generates a tree structure of the given object type and lets you 
% traverse its nodes via visitors. Each node represents one property (or 
% collection item) of the underlying object and tells you exactly if and how 
% the value differs from the base version. It also provides accessors to read, 
% write and remove the value from or to any given instance. This way, all you 
% need to worry about is how to treat changes and not how to find them.
% 
% This library has been battle-tested in a rather big project of mine, where I 
% use it to generate activity streams, resolve database update conflics, 
% display change logs and limit the scope of entity updates to only a subset 
% of properties, based on the context or user permissions. It didn't let me 
% down so far and I hope that it can help you too!
% 
% Use Cases
% 
% Java Object Diff is currently used (but not limited) to...
% 
% Generate Facebook-like activity streams
% Visualize the differences between object versions
% Automatically resolve conflicts on concurrent database updates
% Detect and persist only properties that were actually changed
% Contribute
% 
% You discovered a bug or have an idea for a new feature? Great, why don't you 
% send me a Pull Request (PR) so everyone can benefit from it? To help you 
% getting started, here is a brief guide with everyting you need to know to 
% get involved!
% 
% Donate
% 
% If you’d like to support this project with a small donation, you can do so 
% via Flattr or Bitcoin.
% 
% Alternatively you could send me a nice tweet, start contributing, write a 
% blog post about this project, tell your friends about it or simply star this 
% repository. I'm happy about everything that shows me that someone out there 
% is actually using this library and appreciates all the hard work that goes 
% into its development.
% 
% https://github.com/jpace/diffj
% DiffJ
% DiffJ compares Java files based on their code, without regard to formatting, 
% organization, comments, or whitespace. Differences are reported as to the 
% precise type of change.
% 
% DiffJ was designed for comparing code when refactoring and reformatting, 
% with the differences narrowed to the extent possible, thus isolating 
% changes.
